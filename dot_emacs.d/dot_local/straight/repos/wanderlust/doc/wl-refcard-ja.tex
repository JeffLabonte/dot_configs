% Reference Card for Wanderlust
% The format of this file is adapted from the GNU Emacs reference card
%
% You can compile with luatex and luatexja.sty.

\input luatexja.sty
\pagewidth=297mm
\pageheight=210mm
\pdfvariable horigin 1 true in
\pdfvariable vorigin 1 true in

% Reference Card for GNU Emacs version 20 on Unix systems
%**start of header
\newcount\columnsperpage

% This file can be printed with 1, 2, or 3 columns per page (see below).
% Specify how many you want here.  Nothing else needs to be changed.

\columnsperpage=3

% Copyright (c) 1987, 1993, 1996, 1997 Free Software Foundation, Inc.

% This file is part of GNU Emacs.

% GNU Emacs is free software; you can redistribute it and/or modify
% it under the terms of the GNU General Public License as published by
% the Free Software Foundation; either version 2, or (at your option)
% any later version.

% GNU Emacs is distributed in the hope that it will be useful,
% but WITHOUT ANY WARRANTY; without even the implied warranty of
% MERCHANTABILITY or FITNESS FOR A PARTICULAR PURPOSE.  See the
% GNU General Public License for more details.

% You should have received a copy of the GNU General Public License
% along with GNU Emacs; see the file COPYING.  If not, write to
% the Free Software Foundation, Inc., 59 Temple Place - Suite 330,
% Boston, MA 02111-1307, USA.

% This file is intended to be processed by plain TeX (TeX82).
%
% The final reference card has six columns, three on each side.
% This file can be used to produce it in any of three ways:
% 1 column per page
%    produces six separate pages, each of which needs to be reduced to 80%.
%    This gives the best resolution.
% 2 columns per page
%    produces three already-reduced pages.
%    You will still need to cut and paste.
% 3 columns per page
%    produces two pages which must be printed sideways to make a
%    ready-to-use 8.5 x 11 inch reference card.
%    For this you need a dvi device driver that can print sideways.
% Which mode to use is controlled by setting \columnsperpage above.
%
% Author:
%  Stephen Gildea
%  Internet: gildea@stop.mail-abuse.org
%
% Thanks to Paul Rubin, Bob Chassell, Len Tower, and Richard Mlynarik
% for their many good ideas.

% If there were room, it would be nice to see a section on Dired.

\input version.tex
\def\year{2001}

\def\shortcopyrightnotice{\vskip 1ex plus 2 fill
  \centerline{\small \copyright\ \year\ Yuuichi Teranishi
  Permissions on back.}}

\def\copyrightnotice{
\vskip 1ex plus 2 fill\begingroup\small
\centerline{Copyright \copyright\ \year\  Yuuichi Teranishi}
\centerline{for Wanderlust \versionnumber}

Permission is granted to make and distribute copies of
this card provided the copyright notice and this permission notice
are preserved on all copies.

\endgroup}

% make \bye not \outer so that the \def\bye in the \else clause below
% can be scanned without complaint.
\def\bye{\par\vfill\supereject\end}

\newdimen\intercolumnskip	%horizontal space between columns
\newbox\columna			%boxes to hold columns already built
\newbox\columnb

\def\ncolumns{\the\columnsperpage}

\message{[\ncolumns\space 
  column\if 1\ncolumns\else s\fi\space per page]}

\def\scaledmag#1{ scaled \magstep #1}

% This multi-way format was designed by Stephen Gildea October 1986.
% Note that the 1-column format is fontfamily-independent.
\if 1\ncolumns			%one-column format uses normal size
  \hsize 4in
  \vsize 10in
  \voffset -.7in
  \font\titlefont=\fontname\tenbf \scaledmag3
  \font\headingfont=\fontname\tenbf \scaledmag2
  \font\smallfont=\fontname\sevenrm
  \font\smallsy=\fontname\sevensy

  \footline{\hss\folio}
  \def\makefootline{\baselineskip10pt\hsize6.5in\line{\the\footline}}
\else				%2 or 3 columns uses prereduced size
  \hsize 3.2in
  \vsize 7.95in
  \hoffset -.75in
  \voffset -.745in
  \font\titlefont=cmbx10 \scaledmag2
  \font\headingfont=cmbx10 \scaledmag1
  \font\smallfont=cmr6
  \font\smallsy=cmsy6
  \font\eightrm=cmr8
  \font\eightbf=cmbx8
  \font\eightit=cmti8
  \font\eighttt=cmtt8
  \font\eightmi=cmmi8
  \font\eightsy=cmsy8
  \textfont0=\eightrm
  \textfont1=\eightmi
  \textfont2=\eightsy
  \def\rm{\eightrm}
  \def\bf{\eightbf}
  \def\it{\eightit}
  \def\tt{\eighttt}
  \normalbaselineskip=.8\normalbaselineskip
  \normallineskip=.8\normallineskip
  \normallineskiplimit=.8\normallineskiplimit
  \normalbaselines\rm		%make definitions take effect

  \if 2\ncolumns
    \let\maxcolumn=b
    \footline{\hss\rm\folio\hss}
    \def\makefootline{\vskip 2in \hsize=6.86in\line{\the\footline}}
  \else \if 3\ncolumns
    \let\maxcolumn=c
    \nopagenumbers
  \else
    \errhelp{You must set \columnsperpage equal to 1, 2, or 3.}
    \errmessage{Illegal number of columns per page}
  \fi\fi

  \intercolumnskip=.46in
  \def\abc{a}
  \output={%			%see The TeXbook page 257
      % This next line is useful when designing the layout.
      %\immediate\write16{Column \folio\abc\space starts with \firstmark}
      \if \maxcolumn\abc \multicolumnformat \global\def\abc{a}
      \else\if a\abc
	\global\setbox\columna\columnbox \global\def\abc{b}
        %% in case we never use \columnb (two-column mode)
        \global\setbox\columnb\hbox to -\intercolumnskip{}
      \else
	\global\setbox\columnb\columnbox \global\def\abc{c}\fi\fi}
  \def\multicolumnformat{\shipout\vbox{\makeheadline
      \hbox{\box\columna\hskip\intercolumnskip
        \box\columnb\hskip\intercolumnskip\columnbox}
      \makefootline}\advancepageno}
  \def\columnbox{\leftline{\pagebody}}

  \def\bye{\par\vfill\supereject
    \if a\abc \else\null\vfill\eject\fi
    \if a\abc \else\null\vfill\eject\fi
    \end}  
\fi

% we won't be using math mode much, so redefine some of the characters
% we might want to talk about
\catcode`\^=12
\catcode`\_=12

\chardef\\=`\\
\chardef\{=`\{
\chardef\}=`\}

\hyphenation{mini-buf-fer}

\parindent 0pt
\parskip 1ex plus .5ex minus .5ex

\def\small{\smallfont\textfont2=\smallsy\baselineskip=.8\baselineskip}

% newcolumn - force a new column.  Use sparingly, probably only for
% the first column of a page, which should have a title anyway.
\outer\def\newcolumn{\vfill\eject}

% title - page title.  Argument is title text.
\outer\def\title#1{{\titlefont\centerline{#1}}\vskip 1ex plus .5ex}

% section - new major section.  Argument is section name.
\outer\def\section#1{\par\filbreak
  \vskip 3ex plus 2ex minus 2ex {\headingfont #1}\mark{#1}%
  \vskip 2ex plus 1ex minus 1.5ex}

\newdimen\keyindent

% beginindentedkeys...endindentedkeys - key definitions will be
% indented, but running text, typically used as headings to group
% definitions, will not.
\def\beginindentedkeys{\keyindent=1em}
\def\endindentedkeys{\keyindent=0em}
\endindentedkeys

% paralign - begin paragraph containing an alignment.
% If an \halign is entered while in vertical mode, a parskip is never
% inserted.  Using \paralign instead of \halign solves this problem.
\def\paralign{\vskip\parskip\halign}

% \<...> - surrounds a variable name in a code example
\def\<#1>{{\it #1\/}}

% kbd - argument is characters typed literally.  Like the Texinfo command.
\def\kbd#1{{\tt#1}\null}	%\null so not an abbrev even if period follows

% beginexample...endexample - surrounds literal text, such a code example.
% typeset in a typewriter font with line breaks preserved
\def\beginexample{\par\leavevmode\begingroup
  \obeylines\obeyspaces\parskip0pt\tt}
{\obeyspaces\global\let =\ }
\def\endexample{\endgroup}

% key - definition of a key.
% \key{description of key}{key-name}
% prints the description left-justified, and the key-name in a \kbd
% form near the right margin.
\def\key#1#2{\leavevmode\hbox to \hsize{\vtop
  {\hsize=.75\hsize\rightskip=1em
  \hskip\keyindent\relax#1}\kbd{#2}\hfil}}

\newbox\metaxbox
\setbox\metaxbox\hbox{\kbd{M-x }}
\newdimen\metaxwidth
\metaxwidth=\wd\metaxbox

% metax - definition of a M-x command.
% \metax{description of command}{M-x command-name}
% Tries to justify the beginning of the command name at the same place
% as \key starts the key name.  (The "M-x " sticks out to the left.)
\def\metax#1#2{\leavevmode\hbox to \hsize{\hbox to .75\hsize
  {\hskip\keyindent\relax#1\hfil}%
  \hskip -\metaxwidth minus 1fil
  \kbd{#2}\hfil}}

% threecol - like "key" but with two key names.
% for example, one for doing the action backward, and one for forward.
\def\threecol#1#2#3{\hskip\keyindent\relax#1\hfil&\kbd{#2}\hfil\quad
  &\kbd{#3}\hfil\quad\cr}

%**end of header


\title{Wanderlust Reference Card}

\centerline{(for version \versionnumber)}

\section{Wanderlust の起動}

\key{Wanderlust を起動する}{M-x wl}
\key{メッセージを作成}{M-x wl-draft}

\section{Folder Mode}

\key{Wanderlust をやめる}{q}
\key{次}{n}
\key{前}{p}
\key{次の未読}{N}
\key{前の未読}{P}
\key{指定したフォルダに入る}{g}
\key{このフォルダに入る}{SPC}
\key{グループの開閉}{SPC}
\key{アクセスグループの再スキャン}{C-u SPC}
\key{新規メッセージをチェック}{s}
\key{同期}{S}
\key{プリフェッチ}{I}
\key{全て既読に}{c}
\key{エクスパイア}{e}
\key{メッセージの作成}{w}
\key{メッセージの作成(宛先を推測)}{W}
\key{アドレス帳の再読込}{Z}
\key{ごみ箱を空に}{E}

\section{folder management}

\key{カット}{C-k}
\key{領域をカット}{C-w}
\key{貼り付け}{C-y}
\key{フォルダを加える}{m a}
\key{グループを加える}{m g}
\key{アクセスグループを加える}{m A}

\section{access group}

\key{非講読フォルダも表示}{L}
\key{非講読フォルダを隠す}{l}
\key{講読をやめる}{u}
\key{講読をやめる(領域)}{U}

\newcolumn
\section{Summary Mode}

\key{フォルダバッファに戻る}{q}
\key{指定したフォルダに入る}{g}
\key{次}{n}
\key{前}{p}
\key{次の未読}{N}
\key{前の未読}{P}
\key{先頭のメッセージ}{<}
\key{末尾のメッセージ}{>}
\key{直前に見たメッセージ}{TAB}
\key{メッセージを読む}{SPC}
\key{メッセージの再読込}{C-u .}
\key{メッセージ表示のトグル}{v}
\key{メッセージを作成}{w}
\key{メッセージを作成(宛先を推測)}{W}
\key{返信}{a}
\key{引用して返信}{A}
\key{フォワード}{f}
\key{再編集}{E}
\key{印刷}{\#}
\key{メッセージバッファへ}{j}
\key{アドレス帳の再読込}{Z}

\section{mark command}

\key{処分マークを付ける}{d}
\key{削除マークを付ける}{D}
\key{リファイルマークを付ける}{o}
\key{コピーマークを付ける}{O}
\key{プリフェッチマークを付ける}{i}
\key{再送マークを付ける}{\~{}}
\key{重要マークを付ける}{\$}
\key{標的マークを付ける}{*}
\key{全てのメッセージを標的にする}{m a}
\key{スレッドを標的にする}{m t}
\key{メッセージを選び出して標的に}{?}
\key{標的の中からメッセージを選び出す}{m *}
\key{標的をリファイル}{m o}
\key{標的を削除}{m d}
\key{標的を印刷}{m \#}
\key{標的の各々をパイプに流す}{m |}
\key{一時的マークのアクションを実行}{x}

\section{prefix arguments for marking}

\key{標的に対して}{m}
\key{カーソル以降のスレッドに対して}{t}
\key{領域に対して}{r}

\newcolumn
\section{sticky summary}

\key{サマリをスティッキーに}{M-s}
\key{指定したフォルダに入る}{G}
\key{次のサマリ}{C-c C-n}
\key{前のサマリ}{C-c C-p}
\key{サマリを破棄}{C-u q}

\section{Message Mode}

\key{サマリバッファに戻る}{q}
\key{スクロールか次のパート}{SPC}
\key{スクロールか前のパート}{DEL}
\key{次のパート}{n}
\key{前のパート}{p}
\key{パートを再生}{v}
\key{パートを取り出す}{e}
\key{パートを表示}{C-c C-v C-c}
\key{パートを隠す}{C-c C-d C-c}
\key{MIME-View のヘルプ}{?}

\section{Draft Mode}

\key{メッセージを引用}{C-c C-y}
\key{テンプレートの挿入}{C-c C-j}
\key{署名の挿入}{C-c C-w}
\key{領域を省く}{C-c C-d}
\key{プレビュー}{C-c C-p}
\key{送信}{C-c C-c}
\key{草稿を保存}{C-c C-z}
\key{草稿を破棄}{C-c C-k}
\key{他のドラフトへ}{C-c C-o}
\key{ファイルを添付}{C-c C-x TAB}
%\key{Insert a reference to external body}{C-c C-x C-e}
\key{MIME タグの挿入}{C-c C-x t}
%\key{Insert a text message}{C-c C-x C-t}
\key{メッセージを暗号化}{C-c C-x e}
\key{メッセージに署名}{C-c C-x s}
\key{MIME-Edit のヘルプ}{C-c C-x ?}

\shortcopyrightnotice

\newcolumn
\section{Plugged Status}

\key{プラグ状態のトグル(全体)}{M-t}
\key{プラグドモードへの入退出}{C-t}
\key{プラグ状態のトグル(個別)}{SPC}

\section{Address Book Management}

\key{アドレスマネージャの起動}{C-c C-a}
\key{アドレスマネージャの終了}{q}
\key{{\tt To:} に指定}{t}
\key{{\tt Cc:} に指定}{c}
\key{{\tt Bcc:} に指定}{b}
\key{宛先の指定を解除}{u}
\key{指定に沿ってメッセージを作成}{x}
\key{新規登録}{a}
\key{編集}{e}
\key{削除}{d}

\copyrightnotice

\bye

% Local variables:
% compile-command: "tex refcard"
% End:
